%%%%%%%%%%%%%%%%%%%%%%%%%%%%%%%%%%%%%%%%%%%%%%%%%%%%%%%%%%%%%%%%%%%%%%%%%%%%%%%
%%  
%% 文档类 NJU-Thesis 用户手册
%%
%% 作者:胡海星,starfish (at) gmail (dot) com
%% 项目主页: https://github.com/Haixing-Hu/nju-thesis
%%
%% This file may be distributed and/or modified under the conditions of the
%% LaTeX Project Public License, either version 1.2 of this license or (at your
%% option) any later version. The latest version of this license is in:
%%
%% http://www.latex-project.org/lppl.txt
%%
%% and version 1.2 or later is part of all distributions of LaTeX version
%% 1999/12/01 or later.
%%
%%%%%%%%%%%%%%%%%%%%%%%%%%%%%%%%%%%%%%%%%%%%%%%%%%%%%%%%%%%%%%%%%%%%%%%%%%%%%%%

\documentclass[phd]{njuthesis}
%% njuthesis 文档类的可选参数有:
%%   nobackinfo 取消封二页导师签名信息。注意,按照南大的规定,是需要签名页的。
%%   phd/master/bachelor 选择博士/硕士/学士论文

% 自定义的设置
%%%%%%%%%%%%%%%%%%%%%%%%%%%%%%%%%%%%%%%%%%%%%%%%%%%%%%%%%%%%%%%%%%%%%%%%%%%%%%%
%%  
%% 文档类 NJU-Thesis 用户手册
%%
%% 作者:胡海星,starfish (at) gmail (dot) com
%% 项目主页: https://github.com/Haixing-Hu/nju-thesis
%%
%% This file may be distributed and/or modified under the conditions of the
%% LaTeX Project Public License, either version 1.2 of this license or (at your
%% option) any later version. The latest version of this license is in:
%%
%% http://www.latex-project.org/lppl.txt
%%
%% and version 1.2 or later is part of all distributions of LaTeX version
%% 1999/12/01 or later.
%%
%%%%%%%%%%%%%%%%%%%%%%%%%%%%%%%%%%%%%%%%%%%%%%%%%%%%%%%%%%%%%%%%%%%%%%%%%%%%%%%

% 下面的宏包提供了\BibTeX的logo
\usepackage{dtklogos}

% 提供 framed 环境
\usepackage{framed}

% 使用\verb{text}命令的缩写形式|text|
\MakeShortVerb{\|}

% \std{xxxx}表示标准号
\newcommand{\std}[1]{\texttt{#1}}

% \dangericon 表示警告的图标
\font\manfnt=manfnt
\newcommand*{\dangericon}{\manfnt\char127}

% note 环境表示需特别注意的内容
\newenvironment{note}
{\vskip1.5ex\par\noindent\llap{\dangericon\hskip2mm}\hskip\parindent\textbf{注意:}}
{\vskip1.5ex}

% syntax 环境表示语法描述
\newenvironment{syntax}{\begin{center}}{\end{center}}
\usepackage{fancyvrb}

\DefineVerbatimEnvironment{shell}{Verbatim}%
  {frame=single,framerule=0.1mm,rulecolor=\color{black},%
   framesep=2mm,fontsize=\small}

\DefineVerbatimEnvironment{tex}{Verbatim}%
  {frame=single,framerule=0.1mm,rulecolor=\color{black},%
   framesep=2mm,baselinestretch=1.2,fontsize=\small}

\newcommand{\cs}[1]{\texttt{\textbackslash{}#1}}
\newcommand{\env}[1]{\texttt{#1}}
\newcommand{\meta}[1]{\ensuremath{\langle}\textit{#1}\ensuremath{\rangle}}
\newcommand{\oarg}[1]{\texttt{[}\meta{#1}\texttt{]}}
\newcommand{\marg}[1]{\texttt{\{}\meta{#1}\texttt{\}}}
\newcommand{\parg}[1]{\texttt{(}\meta{#1}\texttt{)}}



%%%%%%%%%%%%%%%%%%%%%%%%%%%%%%%%%%%%%%%%%%%%%%%%%%%%%%%%%%%%%%%%%%%%%%%%%%%%%%%
% 设置《国家图书馆封面》的内容,仅博士论文才需要填写

% 设置论文按照《中国图书资料分类法》的分类编号
\classification{0175.2}
% 论文的密级。需按照GB/T 7156-2003标准进行设置。预定义的值包括:
% - \openlevel,表示公开级:此级别的文献可在国内外发行和交换。
% - \controllevel,表示限制级:此级别的文献内容不涉及国家秘密,但在一定时间内
%   限制其交流和使用范围。
% - \confidentiallevel,表示秘密级:此级别的文献内容涉及一般国家秘密。
% - \clasifiedlevel,表示机密级:此级别的文献内容涉及重要的国家秘密 。
% - \mostconfidentiallevel,表示绝密级:此级别的文献内容涉及最重要的国家秘密。
% 此属性可选,默认为\openlevel,即公开级。
\securitylevel{\openlevel}
% 设置论文按照《国际十进分类法UDC》的分类编号
% 该编号可在下述网址查询:http://www.udcc.org/udcsummary/php/index.php?lang=chi
\udc{004.72}
% 国家图书馆封面上的论文标题第一行,不可换行。此属性可选,默认值为通过\title设置的标题。
\nlctitlea{文档类{\njuthesis}}
% 国家图书馆封面上的论文标题第二行,不可换行。此属性可选,默认值为空白。
\nlctitleb{用户手册}
% 国家图书馆封面上的论文标题第三行,不可换行。此属性可选,默认值为空白。
\nlctitlec{}
% 导师的单位名称及地址
\supervisorinfo{南京大学计算机科学与技术系~~南京市汉口路22号~~210093}
% 答辩委员会主席
\chairman{宋方敏~~教授}
% 第一位评阅人
\reviewera{张三~~教授}
% 第二位评阅人
\reviewerb{李四~~副教授}
% 第三位评阅人
\reviewerc{王五~~教授}
% 第四位评阅人
\reviewerd{赵六~~研究员}

%%%%%%%%%%%%%%%%%%%%%%%%%%%%%%%%%%%%%%%%%%%%%%%%%%%%%%%%%%%%%%%%%%%%%%%%%%%%%%%
% 设置论文的中文封面

% 论文标题,不可换行
\title{文档类{\njuthesis}用户手册}
% 论文作者姓名
\author{胡海星}
% 论文作者联系电话
\telphone{xxxxxxxx}
% 论文作者电子邮件地址
\email{xxxxx@gmail.com}
% 论文作者学生证号
\studentnum{xxxxxxxx}
% 论文作者入学年份(年级)
\grade{xxxx}
% 导师姓名职称
\supervisor{宋方敏~~教授}
% 导师的联系电话
\supervisortelphone{xxxxxxxxx}
% 论文作者的学科与专业方向
\major{计算机软件与理论}
% 论文作者的研究方向
\researchfield{量子算法与量子逻辑}
% 论文作者所在院系的中文名称
\department{计算机科学与技术系}
% 论文作者所在学校或机构的名称。此属性可选,默认值为``南京大学''。
\institute{南京大学}
% 论文的提交日期,需设置年、月、日。
\submitdate{2013年9月5日}
% 论文的答辩日期,需设置年、月、日。
\defenddate{2013年9月20日}
% 论文的定稿日期,需设置年、月、日。此属性可选,默认值为最后一次编译时的日期,精确到日。
\date{2013年8月28日}

%%%%%%%%%%%%%%%%%%%%%%%%%%%%%%%%%%%%%%%%%%%%%%%%%%%%%%%%%%%%%%%%%%%%%%%%%%%%%%%
% 设置论文的英文封面

% 论文的英文标题,注意其中不可换行
\englishtitle{The User Manual of the {\njuthesis} Document Class}
% 论文作者姓名的拼音
\englishauthor{HU Hai-Xing}
% 导师姓名职称的英文
\englishsupervisor{Professor SONG Fang-Min}
% 论文作者学科与专业的英文名
\englishmajor{Computer Software and Theory}
% 论文作者所在院系的英文名称
\englishdepartment{Department of Computer Science and Technology}
% 论文作者所在学校或机构的英文名称。此属性可选,默认值为``Nanjing University''。
\englishinstitute{Nanjing University}
% 论文完成日期的英文形式,它将出现在英文封面下方。需设置年、月、日。日期格式使用美国的日期
% 格式,即``Month day, year'',其中``Month''为月份的英文名全称,首字母大写;``day''为
% 该月中日期的阿拉伯数字表示;``year''为年份的四位阿拉伯数字表示。此属性可选,默认值为最后
% 一次编译时的日期。
\englishdate{Aug 28, 2013}

%%%%%%%%%%%%%%%%%%%%%%%%%%%%%%%%%%%%%%%%%%%%%%%%%%%%%%%%%%%%%%%%%%%%%%%%%%%%%%%
\begin{document}

% 制作国家图书馆封面(博士学位论文才需要)
\makenlctitle
% 制作中文封面
\maketitle
% 制作英文封面
\makeenglishtitle
% 开始前言部分
\frontmatter
% 论文的中文摘要
\include{chinese-abstract}
% 论文的英文摘要
%%%%%%%%%%%%%%%%%%%%%%%%%%%%%%%%%%%%%%%%%%%%%%%%%%%%%%%%%%%%%%%%%%%%%%%%%%%%%%%
%%  
%% 文档类 NJU-Thesis 用户手册
%%
%% 作者:胡海星,starfish (at) gmail (dot) com
%% 项目主页: https://github.com/Haixing-Hu/nju-thesis
%%
%% This file may be distributed and/or modified under the conditions of the
%% LaTeX Project Public License, either version 1.2 of this license or (at your
%% option) any later version. The latest version of this license is in:
%%
%% http://www.latex-project.org/lppl.txt
%%
%% and version 1.2 or later is part of all distributions of LaTeX version
%% 1999/12/01 or later.
%%
%%%%%%%%%%%%%%%%%%%%%%%%%%%%%%%%%%%%%%%%%%%%%%%%%%%%%%%%%%%%%%%%%%%%%%%%%%%%%%%


% 设置论文的英文摘要
% 设置英文摘要页面的论文标题及副标题的第一行。
% 此属性可选,其默认值为使用|\englishtitle|命令所设置的论文标题
%% \englishabstracttitlea{}
% 设置英文摘要页面的论文标题及副标题的第二行。
% 此属性可选,其默认值为空白
%% \englishabstracttitleb{}
% 论文的英文摘要
\begin{englishabstract}
This document is the user manual of the {\XeLaTeX} document class for
typesetting the degree thesis of Nanjing University. The document describes the
installation and usage of the {\njuthesis} document class in details. Since the
document itself is typeset by the {\njuthesis} document class, it could also be
used as a usage example and starting template for the document class.

% 英文关键词。关键词之间用英文半角逗号隔开,末尾无符号。
\englishkeywords{degree thesis, typesetting, {\XeLaTeX}, Nanjing Univeristy}
\end{englishabstract}

% 论文的前言,应放在目录之前,中英文摘要之后
\include{preface}
% 生成论文目录
\tableofcontents
% 生成表格目录。如无需表格目录则可注释掉下述语句。
\listoftables
% 生成插图目录。如无需插图目录则可注释掉下述语句。
\listoffigures
%%%%%%%%%%%%%%%%%%%%%%%%%%%%%%%%%%%%%%%%%%%%%%%%%%%%%%%%%%%%%%%%%%%%%%%%%%%%%%%
% 开始正文部分
\mainmatter
% 下面是正文的各章节
\chapter{绪论}\label{chapter:introduction}

\section{引言}

\section{学位论文排版的要求}

\section{现有的解决方案}

\section{现有方案的不足}

\section{本工作的意义和价值}

%%%%%%%%%%%%%%%%%%%%%%%%%%%%%%%%%%%%%%%%%%%%%%%%%%%%%%%%%%%%%%%%%%%%%%%%%%%%%%%
%%  
%% 文档类 NJU-Thesis 用户手册
%%
%% 作者:胡海星,starfish (at) gmail (dot) com
%% 项目主页: https://github.com/Haixing-Hu/nju-thesis
%%
%% This file may be distributed and/or modified under the conditions of the
%% LaTeX Project Public License, either version 1.2 of this license or (at your
%% option) any later version. The latest version of this license is in:
%%
%% http://www.latex-project.org/lppl.txt
%%
%% and version 1.2 or later is part of all distributions of LaTeX version
%% 1999/12/01 or later.
%%
%%%%%%%%%%%%%%%%%%%%%%%%%%%%%%%%%%%%%%%%%%%%%%%%%%%%%%%%%%%%%%%%%%%%%%%%%%%%%%%

\chapter{文档类的安装}\label{chapter:installtion}

\section{\TeX 系统的安装}

\section{\njuthesis 的安装}

\section{\njuthesis 的测试}

\section{\njuthesis 的升级}

\chapter{学位论文的排版}

\section{\LaTeX 排版}


\section{学位论文的结构}


\section{使用文档类}

\subsection{元信息}

\subsection{封面}

\subsection{摘要}

\subsection{前言}

\subsection{目录}

\subsection{插图目录}

\subsection{表格目录}

\subsection{缩写和符号列表}

\subsection{术语表}

\subsection{正文}

\subsubsection{章、节}

\subsubsection{插图}

\subsubsection{表格}

\subsubsection{公式}

\subsubsection{定理}

\subsubsection{证明}

\subsubsection{算法}

\subsubsection{交叉引用}

\subsubsection{脚注}

\subsubsection{引文}

\subsection{致谢}

\subsection{参考文献列表}

\subsection{附录}

\subsection{索引}

\subsection{作者简历}

\subsection{出版授权}




\chapter{参考文献数据库}\label{chap:bib}

%%%%%%%%%%%%%%%%%%%%%%%%%%%%%%%%%%%%%%%%%%%%%%%%%%%%%%%%%%%%%%%%%%%%%%%%%%%%%%%
\section{简介}\label{sec:bib-intro}


%%%%%%%%%%%%%%%%%%%%%%%%%%%%%%%%%%%%%%%%%%%%%%%%%%%%%%%%%%%%%%%%%%%%%%%%%%%%%%%
\section{格式}\label{sec:bib-format}


%%%%%%%%%%%%%%%%%%%%%%%%%%%%%%%%%%%%%%%%%%%%%%%%%%%%%%%%%%%%%%%%%%%%%%%%%%%%%%%

\section{文献项字段}\label{sec:bib-field}

\subsection{language}

|language|字段用于说明该文献项对应的文献所使用的语言。

\begin{note}
如果是外文文献,此字段可忽略;如果是中文文献,此字段\emph{必须}填写|cn|或|zh|。
\end{note}

{\BibTeX}需要根据此字段判别文献的语言,对于中文文献和外文文献采用不同的排版方式。

\subsection{author}

|author|字段表示文献的作者。

|author|字段可以有多个作者名。多个作者名之间\emph{必须}用英文词``and''隔开。根据
\cite{gbt7714-2005}的规定,{\BibTeX}会自动只使用前三个作者名进行排版,若还有多余作者,
会在作者名之后加``, 等''(对于中文文献)或``, et al''(对于外文文献)。

\beign{note}
即使文献作者名都是中文名,多个作者名之间也\emph{必须}用英文词``and''隔开。
\end{note}

外文作者名建议按照\cite[157]{lamport1994latex}中的要求填写,{\BibTeX}系统会自动将其
转换为符合\cite{gbt7714-2005}规范的格式。

中文作者名采用姓在前名在后的方式填写,姓和名之间不需加空格。

外文作者的中文翻译名必须按照\cite{gbt7714-2005}中的要求,要么只有其姓的中文译名,
要么可选择在其姓的中文译名后增加其名的缩写字母,例如:``昂温 P S'',缩写名后面
的``.''可以省略。

\begin{note}
对于中文文献,若其是外文文献的翻译版,应将其作者名写成中文译名。翻译者的姓名应使
用|translator|字段指定。
\end{note}

\section{editor}

|editor|字段表示文献的编辑者。其具体要求和|author|字段类似。

|editor|字段和|author|字段的区别在于:|author|是所引用的文献的主要负责人,
|editor|则是所引用文献本身,或其所在的合集的编辑。例如:
\begin{itemize}
\item 如果所文献是一本书,但每个章节由不同的作者所写,而整本书由某个编辑者编辑。此时,
  \begin{itemize}
  \item 如果论文中引用的是全书,则应该将该文献作为|book|类型,其|editor|字段为编辑
    者,|author|字段不填。
  \item 如果论文中引用的是全书中由某个作者写的某个章节,则应该将文献作为
    |inbook|或|incollection|类型(取决于该章节是否有自己的标题,详见
    \autoref{sec:bib-type}的解释),其|author|字段为该章节作者,|editor|字段为
    书籍的编辑者。
  \end{itemize}
\item 如果文献是会议论文集,此时:
  \begin{itemize}
  \item 如果论文引用的是论文集全集,则应该将文献作为|proceedings|类型,其|editor|字段
    为该论文集的编辑,|author|字段不填;
  \item 如果论文引用的是论文集中的某篇文章,则应该将文献作为|inproceedings|类型,
    其|author|字段为该文章作者,|editor|字段为该论文集的编辑者。
  \end{itemize}
\end{itemize}

关于|editor|和|author|字段的详细解释,请参见

%%%%%%%%%%%%%%%%%%%%%%%%%%%%%%%%%%%%%%%%%%%%%%%%%%%%%%%%%%%%%%%%%%%%%%%%%%%%%%%
\section{文献类型}\label{sec:bib-type}

\subsection{书籍}

书籍类文献所对应的{\BibTeX}文献项类型为|book|。

该类型文献的其必需字段包括:|author|、|title|、|publisher|、|year|。

其可选字段包括:|editor|、|translator|、|edition|、|volume|、|number|、|series|、
|address|、|url|、|doi|。

其他字段将不起作用。

\subsection{专著中的析出文献}

\subsection{期刊论文}

\subsection{会议论文}

\subsection{技术报告}

\subsection{学位论文}

\subsection{未发表论文}

\subsection{手册}

\subsection{标准}

\subsection{专利}

\subsection{网页}

%%%%%%%%%%%%%%%%%%%%%%%%%%%%%%%%%%%%%%%%%%%%%%%%%%%%%%%%%%%%%%%%%%%%%%%%%%%%%%%

\chapter{学位论文的编译}

\section{简介}

\section{编译命令}

\section{常见错误及解决方法}

%%%%%%%%%%%%%%%%%%%%%%%%%%%%%%%%%%%%%%%%%%%%%%%%%%%%%%%%%%%%%%%%%%%%%%%%%%%%%%%
%%  
%% 文档类 NJU-Thesis 用户手册
%%
%% 作者:胡海星,starfish (at) gmail (dot) com
%% 项目主页: https://github.com/Haixing-Hu/nju-thesis
%%
%% This file may be distributed and/or modified under the conditions of the
%% LaTeX Project Public License, either version 1.2 of this license or (at your
%% option) any later version. The latest version of this license is in:
%%
%% http://www.latex-project.org/lppl.txt
%%
%% and version 1.2 or later is part of all distributions of LaTeX version
%% 1999/12/01 or later.
%%
%%%%%%%%%%%%%%%%%%%%%%%%%%%%%%%%%%%%%%%%%%%%%%%%%%%%%%%%%%%%%%%%%%%%%%%%%%%%%%%

\chapter{结论}


% 致谢章节
\include{acknowledgement}
% 附录
\appendix

\chapter{南京大学博士(硕士)学位论文编写格式规定(试行)}\label{chap:njureq}

\section{适用范围}

本规定适用于博士学位论文编写,硕士学位论文编写应参照执行。

\section{引用标准}

GB7713科学技术报告、学位论文和学术论文的编写格式。

GB7714文后参考文献著录规则。

\section{印制要求}

论文必须用白色纸印刷,并用A4(210mm×297mm)标准大小的白纸。纸的四周应留足空白
边缘,上方和左侧应空边25mm以上,下方和右侧应空边20mm以上。除前置部分外,其它
部分双面印刷。

论文装订不要用铁钉,以便长期存档和收藏。

论文封面与封底之间的中缝(书脊)必须有论文题目、作者和学校名。

\section{编写格式}

论文由前置部分、主体部分、附录部分(必要时)、结尾部分(必要时)组成。

前置部分包括封面,题名页,声明及说明,前言,摘要(中、英文),关键词,目次页,
插图和附表清单(必要时),符号、标志、缩略词、首字母缩写、单位、术语、名词解释
表(必要时)。

主体部分包括绪论(作为正文第一章)、正文、结论、致谢、参考文献表。

附录部分包括必要的各种附录。

结尾部分包括索引和封底。

\section{前置部分}

\subsection{封面(博士论文国图版用)}

封面是论文的外表面,提供应有的信息,并起保护作用。

封面上应包括下列内容:
\begin{enumerate}
\item 分类号  在左上角注明分类号,便于信息交换和处理。一般应注明《中国图书资
  料分类法》的类号,同时应注明《国际十进分类法UDC》的类号;
\item 密级  在右上角注明密级;
\item “博士学位论文”用大号字标明;
\item 题名和副题名   用大号字标明;
\item 作者姓名;
\item 学科专业名称;
\item 研究方向;
\item 导师姓名,职称;
\item 日期包括论文提交日期和答辩日期;
\item 学位授予单位。
\end{enumerate}

\subsection{题名}

题名是以最恰当、最简明的词语反映论文中最重要的特定内容的逻辑组合。

题名所用每一词语必须考虑到有助于选定关键词和编写题录、索引等二次文献可以提供
检索的特定实用信息。

题名应避免使用不常见的缩略词、首字母缩写字、字符、代号和公式等。

题名一般不宜超过20字。

论文应有外文题名,外文题名一般不宜超过10个实词。

可以有副题名。

题名在整本论文中不同地方出现时,应完全相同。

\subsection{前言}

前言是作者对本论文基本特征的简介,如论文背景、主旨、目的、意义等并简述本论文
的创新性成果。

\subsection{摘要}

摘要是论文内容不加注释和评论的简单陈述。

论文应有中、英文摘要,中、英文摘要内容应相同。

摘要应具有独立性和自含性,即不阅读论文的全文,便能获得必要的信息,摘要
中有数据、有结论,是一篇完整的短文,可以独立使用,可以引用,可以用于推广。摘
要的内容应包括与论文同等量的主要信息,供读者确定有无必要阅读全文,也供文摘等
二次文献引用。摘要的重点是成果和结论。

中文摘要一般在1500字,英文摘要不宜超过1500实词。

摘要中不用图、表、化学结构式、非公知公用的符号和术语。

\subsection{关键词}

关键词是为了文献标引工作从论文中选取出来用于表示全文主题内容信息款目的单词或
术语。

每篇论文选取3-8个词作为关键词,以显著的字符另起一行,排在摘要的左下方。在英
文摘要的左下方应标注与中文对应的英文关键词。

\subsection{目次页}

目次页由论文的章、节、附录等的序号、名称和页码组成,另页排在摘要的后面。

\subsection{插图和附表清单}

论文中如图表较多,可以分别列出清单并置于目次页之后。

图的清单应有序号、图题和页码。表的清单应有序号、表题和页码。

符号、标志、缩略词、首字母缩写、计量单位、名词、术语等的注释表符号、标志、缩略词、
首字母缩写、计量单位、名词、术语等的注释说明汇集表,应置于图表清单之后。

\section{主体部分}

\subsection{格式}

主体部分由绪论开始,以结论结束。主体部分必须由另页右页开始。每一章必须另页开
始。全部论文章、节、目的格式和版面安排要划一,层次清楚。

\subsection{序号}

论文的章可以写成:第一章。节及节以下均用阿拉伯数字编排序号,如
1.1,1.1.1等。

论文中的图、表、附注、参考文献、公式、算式等一律用阿拉伯数字分别分章依序连续编排
序号。其标注形式应便于互相区别,一般用下例:图1.2;表2.3;附注1);文献[4];式
  (6.3)等。

论文一律用阿拉伯数字连续编页码。页码由首页开始,作为第1页,并为右页另页。封页、
封二、封三和封底不编入页码,应为题名页、前言、目次页等前置部分单独编排页码。页码
必须标注在每页的相同位置,便于识别。

附录依序用大写正体A、B、C、$\cdots$编序号,如:附录A。附录中的图、表、式、参考文
献等另行编序号,与正文分开,也一律用阿拉伯数字编码,但在数码前题以附条序码,如图
A.1;表B.2;式(B.3);文献[A.5]等。

\subsection{绪论}

绪论(综述):简要说明研究工作的目的、范围、相关领域的前人工作和知识空白、理
论基础和分析,研究设想、研究方法和实验设计、预期结果和意义等。一般在教科书中
有的知识,在绪论中不必赘述。

绪论的内容应包括论文研究方向相关领域的最新进展、对有关进展和问题的评价、本论
文研究的命题和技术路线等;绪论应表明博士生对研究方向相关的学科领域有系统深入
的了解,论文具有先进性和前沿性;

为了反映出作者确已掌握了坚实的基础理论和系统的专门知识,具有开阔的科学视野,对研
究方案作了充分论证,绪论应单独成章,列为第一章,绪论的篇幅应达$1\sim 2$万字,不
得少于$1$万字;绪论引用的文献应在$100$篇以上,其中外文文献不少于$60\%$;引用文献
应按正文中引用的先后排列。

\subsection{正文}

论文的正文是核心部分,占主要篇幅。正文必须实事求是,客观真切,准确完备,合乎
逻辑,层次分明,简便可读。

正文的每一章(除绪论外)应有小结,在小结中应明确阐明作者在本章中所做的工作,特
别是创新性成果。凡本论文要用的基础性内容或他人的成果不应单独成章,也不应作过
多的阐述,一般只引结论、使用条件等,不作推导。

\subsubsection{图}

图包括曲线图、构造图、示意图、图解、框图、流程图、记录图、布置图、地图、照片
、图版等。

图应具有“自明性”,即只看图、图题和图例,不阅读正文,就可以理解图意。

图应编排序号。每一图应有简短确切的图题,连同图号置于图下。必要时,应将图上的
符号、标记、代码,以及实验条件等,用最简练的文字,横排于图题下方,作为图例说
明。

曲线图的纵、横坐标必须标注“量、标准规定符号、单位”。此三者只有在不必要标明
(如无量纲等)的情况下方可省略。坐标上标注的量的符号和缩略词必须与正文一致。

照片图要求主题和主要显示部分的轮廓鲜明,便于制版。如用放大缩小的复制品,必须
清晰,反差适中。照片上应该有表示目的物尺寸的标度。

\subsubsection{表}

表的编排,一般是内容和测试项目由左至右横读,数据依序竖排。表应有自明性。

表应编排序号。

每一表应有简短确切的表题,连同标号置于表上。必要时,应将表中的符号、标记、代
码,以及需要说明事项,以最简练的文字,横排于表题下,作为表注,也可以附注于表
下。表内附注的序号宜用小号阿拉伯数字并加圆括号置于被标注对象的右上角,如:
xxx${}^{1)}$;不宜用“*”,以免与数学上共轭和物质转移的符号相混。

表的各栏均应标明“量或测试项目、标准规定符号、单位”。只有在无必要标注的情况下
方可省略。表中的缩略词和符号,必须与正文中一致。

表内同一栏的数字必须上下对齐。表内不宜用“同上”,“同左”和类似词,一律填入具体数字
或文字。表内“空白”代表未测或无此项,“-”或“\textellipsis”(因“-”可能与代表阴性
  反应相混)代表未发现,“0”代表实测结果确为零。

如数据已绘成曲线图,可不再列表。

\subsubsection{数学、物理和化学式}

正文中的公式、算式或方程式等应编排序号,序号标注于该式所在行(当有续行时,应
标注于最后一行)的最右边。

较长的式,另行居中横排。如式必须转行时,只能在$+$,$-$,$\times$,$\div$,$<$,
$>$处转行。上下式尽可能在等号“$=$”处对齐。

小数点用“$.$”表示。大于$999$的整数和多于三位数的小数,一律用半个阿拉伯数字符的小
间隔分开,不用千位撇。对于纯小数应将$0$列于小数点之前。

示例:应该写成$94\ 652.023\ 567$和$0.314\ 325$, 不应写成$94,652.023,567$和
$.314,325$。

应注意区别各种字符,如:拉丁文、希腊文、俄文、德文花体、草体;罗马数字和阿拉伯数
字;字符的正斜体、黑白体、大小写、上下脚标(特别是多层次,如“三踏步”)、上下偏差
等。

\subsubsection{计量单位}

报告、论文必须采用国务院发布的《中华人民共和国法定计量单位》,并遵照《中华人
民共和国法定计量单位使用方法》执行。使用各种量、单位和符号,必须遵循附录B所
列国家标准的规定执行。单位名称和符号的书写方式一律采用国际通用符号。

\subsubsection{符号和缩略词}

符号和缩略词应遵照国家标准的有关规定执行。如无标准可循,可采纳本学科或本专业
的权威性机构或学术团体所公布的规定;也可以采用全国自然科学名词审定委员会编印
的各学科词汇的用词。如不得不引用某些不是公知公用的、且又不易为同行读者所理解
的、或系作者自定的符号、记号、缩略词、首字母缩写字等时,均应在第一次出现时一
一加以说明,给以明确的定义。

\subsection{结论}

报告、论文的结论是最终的、总体的结论,不是正文中各段的小结的简单重复。结论应
该准确、完整、明确、精炼。在结论中要清楚地阐明论文中有那些自己完成的成果,特
别是创新性成果;

如果不可能导出应有的结论,也可以没有结论而进行必要的讨论。可以在结论或讨论中
提出建议、研究设想、仪器设备改进意见、尚待解决的问题等。

\subsection{致谢}

可以在正文后对下列方面致谢:

\begin{itemize}
\item 国家科学基金、资助研究工作的奖学金基金、合作单位、资助或支持的企业、组织或个
人;
\item 协助完成研究工作和提供便利条件的组织或个人;
\item 在研究工作中提出建议和提供帮助的人;
\item 给予转载和引用权的资料、图片、文献、研究思想和设想的所有者;
\item 其他应感谢的组织或个人。
\end{itemize}

\subsection{参考文献表}

\subsubsection{专著著录格式}

主要责任者,其他责任者,书名,版本,出版地:出版者,出版年

例:1. 刘少奇,论共产党员的修养,修订2版,北京:人民出版社,1962

\subsubsection{连续出版物中析出的文献著录格式}

析出文献责任者,析出文献其他责任者,析出题名,原文献题名,版本:文献中的位置。

例:2. 李四光,地壳构造与地壳运动,中国科学,1973 (4):400-429

参考文献采用顺序编码制,按论文正文所引用文献出现的先后顺序连续编码。

\section{附录}

附录是作为报告、论文主体的补充项目,并不是必需的。

下列内容可以作为附录编于报告、论文后,也可以另编成册;

\begin{enumerate}
\item 为了整篇论文材料的完整,但编入正文又有损于编排的条理和逻辑性,这一材料
包括比正文更为详尽的信息、研究方法和技术更深入的叙述,建议可以阅读的参考文献
题录,对了解正文内容有用的补充信息等;
\item 由于篇幅过大或取材于复制品而不便于编入正文的材料;
\item 不便于编入正文的罕见珍贵资料;
\item 对一般读者并非必要阅读,但对本专业同行有参考价值的资料;
\item 某些重要的原始数据、数学推导、计算程序、框图、结构图、注释、统计表、计
算机打印输出件等。
\end{enumerate}

附录与正文连续编页码。

每一附录均另页起。

\section{结尾部分 (必要时)}

为了将论文迅速存储入电子计算机,可以提供有关的输入数据。可以编排分类索引、著者索
引、关键词索引等。

\chapter{“目次”和“目录”的区别}

通常我们都将书籍的章节列表称之为“目录”,但根据国家标准《GB/T 7713.1-2006: 学位论文编写规则》
\cite{gbt7713.1-2006}和《GB/T 13417-2009: 期刊目次表》\cite{gbt13417-2009},
以及《南京大学博士(硕士)学位论文编写格式规定(试行)》(参见附录\ref{chap:njureq})中的要求,
论文的章节列表其实应该称为“目次”。

那么“目次”究竟是什么意思?它和“目录”有何区别?

\section{百度百科的解释}

对于目录与目次的区别,百度百科上说明如下\cite{baidu2013muci}:

\begin{quotation}
Contents 目次

Table of Contents 目录

目次是目录的排序,目录是内容章节的具体名称。

目录专著始于刘向父子,到清代 《四库全书总目提要》里按经,史,子,集四类编目,每
一大类又分若干小类,类下分子目,大类前有总序,每一小类前有小序,子目后有案语,序
及案语是用来简述著作源流以及分类理由的。而每小类的后面还附有“存目” ,这存目就是
所谓的目次了 。目录又称为书目,是记录图书的书目名称著者,搜藏与流传情况,内容提
要,评价,真伪辨析等内容的。
\end{quotation}

不过百度百科的解释明显不清楚,且有错误。英语单词``content''在Merriam-Webster
在线版词典中作为名词的解释如下:
\begin{quotation}
a: something contained - usually used in plural;\\
b: the topics or matter treated in a written work;\\
c: the principal substance (as written matter, illustrations, 
or music) offered by a World Wide Web site.
\end{quotation}

显然后两个含义都是第一个含义的引申,即``content''作为名词其本意应该是指“内部包含
之物(something contained)”,后引申为文章的“内容”。而``table of contents''
则有“内容列表”的含义,和“目录”或“目次”相关。

\section{程千帆先生的意见}

程千帆先生反对用“目录”这个词。他说:“我写书时,对于底下的篇目我不用目录两个字的,
因为目是目,录是录,我总是写作目次,写篇目也可以,无论如何不能写目录。”\cite{cheng2008}

程先生的话与传统的目录学知识相关。刘向的《七略》是我国古代最早的全国综合性目录,
虽然今已失传,但由于《汉书·艺文志》是根据《七略》编写的,因此仍然可以知道其大概
体例。

\section{“目录”一词的由来}

《汉书·艺文志》中记载道:

\begin{quotation}
昔仲尼没而微言绝,七十子丧而大义乖。故《春秋》分为五,《诗》分为四,《易》有数家
之传。战国从衡,真伪分争,诸子之言纷然殽乱。至秦患之,乃燔灭文章,以愚黔首。汉兴,
改秦之败,大收篇籍,广开献书之路。迄孝武世,书缺简脱,礼坏乐崩,圣上喟然而称曰:
“朕甚闵焉!”于是建藏书之策,置写书之官,下及诸子传说,皆充秘府。至成帝时,以书颇
散亡,使谒者陈农求遗书于天下。诏光禄大夫刘向校经传诸子诗赋,步兵校尉任宏校兵书,
太史令尹咸校数术,侍医李柱国校方技。每一书已,向辄条其篇目,撮其指意,录而奏之。
会向卒,哀帝复使向子侍中奉车都尉歆卒父业。歆于是总群书而奏其《七略》,故有《辑略》,
有《六艺略》,有《诸子略》,有《诗赋略》,有《兵书略》,有《术数略》,有
《方技略》。今删其要,以备篇辑。
\end{quotation}

其中所说的“每一书已,向辄条其篇目,撮其指意,录而奏之”便是“目录”一词最初的用意了。
即“目”指“条其篇目”,“录”指“条其篇目,撮其指意,录而奏之”。这里采用的是余嘉锡先生在
《目录学发微·何谓目录》中的意见,也就是说“目”单指篇目,而“录”是包含篇目与内容大
意两部分内容的。而后由于袭用的缘故,“录”反而隶属在“目”之下了,于是有篇目而无内容
大意的也可称之为“目录”。再往后,只有书名而无篇目的也可称为“目录”。

《昭明文选·任彦开为范始兴求立太宰碑表(李善注)》引刘欲《七略》称:“《尚书》有青丝
编目录”,可知刘向校书,即已使用“口录”一词。班因《汉书·叙传》中,亦有“爱着目录,略序
洪烈,述艺文志第十”之句,表明早在汉代,“目录”二字已作为一个名词而被加以使用。

\section{“目录”一词的含义}

目录:是指著录一批相关文献,并按照一定次序编排而成的揭示与报道文献的工具。它是联系
文献与用户的桥梁和纽带。是书籍文章的缩影。

目录是目和录的总称。“目”指篇名或书名,“录”是对“目”的说明和编次。前人把“目”与“录”
编在一起,谓之“目录”。

\section{“目录”一词的其他含义}

在现代计算机技术术语中,常常把文件系统中的“directory”称为“目录”,表示
在文件系统树中的非叶子节点。这种用法给“目录”一词带来了新的含义。

\section{结论}

根据“目录”一词的由来,及程千帆先生的意见,我们决定遵循国家标准
《GB/T 7713.1-2006: 学位论文编写规则》\cite{gbt7713.1-2006}和
《GB/T 13417-2009: 期刊目次表》\cite{gbt13417-2009},以及
附录\ref{chap:njureq}中的要求,将``table of contents''称为
“目次”而非“目录”。



\chapter{姓名的格式}\label{chap:names}

\section{中国人的中文姓名}

\section{中国人的英文姓名}

\section{外国人的外文姓名}

\section{外国人名的中文译名}


%%%%%%%%%%%%%%%%%%%%%%%%%%%%%%%%%%%%%%%%%%%%%%%%%%%%%%%%%%%%%%%%%%%%%%%%%%%%%%%
% 论文附件
\backmatter
% 令所有未被引用的参考文献也出现在参考文献列表中
\nocite{*}
% 参考文献目录
\bibliography{njuthesis-manual}
% 作者简历与科研成果页,应放在参考文献之后
\include{resume}
% 生成《学位论文出版授权书》页面,应放在最后一页
\makelicense
%%%%%%%%%%%%%%%%%%%%%%%%%%%%%%%%%%%%%%%%%%%%%%%%%%%%%%%%%%%%%%%%%%%%%%%%%%%%%%%
\end{document}
