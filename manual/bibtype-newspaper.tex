%%%%%%%%%%%%%%%%%%%%%%%%%%%%%%%%%%%%%%%%%%%%%%%%%%%%%%%%%%%%%%%%%%%%%%%%%%%%%%%
%%  
%% 文档类 NJU-Thesis 用户手册
%%
%% 作者:胡海星,starfish (at) gmail (dot) com
%% 项目主页: https://github.com/Haixing-Hu/nju-thesis
%%
%% This file may be distributed and/or modified under the conditions of the
%% LaTeX Project Public License, either version 1.2 of this license or (at your
%% option) any later version. The latest version of this license is in:
%%
%% http://www.latex-project.org/lppl.txt
%%
%% and version 1.2 or later is part of all distributions of LaTeX version
%% 1999/12/01 or later.
%%
%%%%%%%%%%%%%%%%%%%%%%%%%%%%%%%%%%%%%%%%%%%%%%%%%%%%%%%%%%%%%%%%%%%%%%%%%%%%%%%

\subsection{报纸}\label{subsec:bibtype-newspaper}

报纸是连续出版物的一种,它与期刊类似。它所对应的{\BibTeX}文献项类型为|newspaper|;对应的
\std{GB/T 3469-1983}文献类型单字码为|N|\cite{gbt3469-1983}。

|newspaper|不是{\BibTeX}的标准类型,而是{\njuthesis}的扩展。之所以将|newspaper|和
|periodical|区分开,是因为期刊(|periodical|)所对应的文献类型单字码为|J|,而报纸
(|newspaper|)所对应的文献类型单字码为|N|\cite{gbt3469-1983}。

\begin{note}
如果引文是报纸的某一期或某几期,则应使用|newspaper|类型;如果引文是某一期报纸中的文章,
请使用|news|类型,具体参见\ref{subsec:bibtype-news}。
\end{note}

\subsubsection{必需字段}

\begin{itemize}
\item |title|:表示该报纸的标题,其格式参见\ref{subsec:bibfield-title};
\item |publisher|:表示该报纸的出版社,其格式参见\ref{subsec:bibfield-publisher};
\item |address|:表示该报纸的出版地,其格式参见\ref{subsec:bibfield-address};
\item |year|表示所引用的报纸的出版年,或所引用的一系列连续报纸的出版年范围,其格式
  参见\ref{subsec:bibfield-year}。
\end{itemize}

\subsubsection{可选字段}

\begin{itemize}
\item |author|:表示该报纸的发行机构或编辑,其格式参见\ref{subsec:bibfield-author};
\item |volume|: 表示所引用的报纸的卷号,或所引用的一系列连续报纸的卷号范围,其格式参
  见\ref{subsec:bibfield-volume};
\item |number|: 表示所引用的报纸的期号,或所引用的一系列连续报纸的期号范围,其格式参
  见\ref{subsec:bibfield-number};
\item |citedate|:表示所引用的报纸的在线版本的引用日期,其格式参见\ref{subsec:bibfield-citedate};
\item |url|:表示该报纸的在线版本的引用URL,其格式参见\ref{subsec:bibfield-url};
\item |doi|:表示该报纸的电子版的DOI编号,其格式参见\ref{subsec:bibfield-doi};
\item |language|:表示该报纸的语言,其格式参见\ref{subsec:bibfield-language};
\item 其他字段将不起作用。
\end{itemize}

\begin{note}
若该报纸对应的文献项有|url|或|doi|字段,则其必须也有|citedate|字段。
若该报纸的语言为中文,则必须将|language|字段设置为|zh|;否则可忽略|language|字段。
\end{note}

\subsubsection{例子}

\begin{verbatim}
@newspaper{financialtimes,
  title={The Financial Times},
  year={1888-1913},
  volume={1-512},
  number={1-1210}
  address={London},
  publisher={Pearson PLC},  
}

@newspaper{renminribao,
  title={人民日报},
  year={2011},
  volume={22892},
  address={北京},
  publisher={人民日报出版社},
  language={zh},
}
\end{verbatim}
