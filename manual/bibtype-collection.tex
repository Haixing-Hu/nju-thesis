%%%%%%%%%%%%%%%%%%%%%%%%%%%%%%%%%%%%%%%%%%%%%%%%%%%%%%%%%%%%%%%%%%%%%%%%%%%%%%%
%%  
%% 文档类 NJU-Thesis 用户手册
%%
%% 作者:胡海星,starfish (at) gmail (dot) com
%% 项目主页: https://github.com/Haixing-Hu/nju-thesis
%%
%% This file may be distributed and/or modified under the conditions of the
%% LaTeX Project Public License, either version 1.2 of this license or (at your
%% option) any later version. The latest version of this license is in:
%%
%% http://www.latex-project.org/lppl.txt
%%
%% and version 1.2 or later is part of all distributions of LaTeX version
%% 1999/12/01 or later.
%%
%%%%%%%%%%%%%%%%%%%%%%%%%%%%%%%%%%%%%%%%%%%%%%%%%%%%%%%%%%%%%%%%%%%%%%%%%%%%%%%

\subsection{汇编}\label{subsec:bibtype-collection}

汇编是指汇总编辑而成的专著\cite{zdic2013huibian},例如关于某一主题的文章的合集,
或同一作者的文献集合之类。汇编包括论文集等\cite{gbt3469-1983}。它所对应的{\BibTeX}文献
项类型为|collection|;所对应的\std{GB/T 3469-1983}文献类型单字码为|G|\cite{gbt3469-1983}。

\begin{note}
“论文集”应使用|collection|类型;而“会议录”和“会议论文集”应使用|proceedings|
或|conference|类型,参见\ref{subsec:bibtype-proceedings}。
\end{note}

\begin{note}
注意要区分“汇编”和“书籍”。一个简单的办法是看其结构。如果该文献是由章节构成,
则该文献属于“书籍”;即使每章作者不同,该文献也属于拥有多个作者的书籍。如果
该文献没有章节结构,而是由多篇独立的文献构成,每篇独立的文献都有自己的独立标题,
则该文献属于“汇编”;即使每篇独立文献的作者都相同,该文献也属于“汇编”。
\end{note}

\subsubsection{必需字段}

\begin{itemize}
\item |author|:表示该汇编的作者,其格式参见\ref{subsec:bibfield-author};
\item |editor|:表示该汇编的编辑者,其格式参见\ref{subsec:bibfield-editor};
\item |title|:表示该汇编的标题,其格式参见\ref{subsec:bibfield-title};
\item |publisher|:表示该汇编的出版社,其格式参见\ref{subsec:bibfield-publisher};
\item |address|:表示该汇编的出版地,其格式参见\ref{subsec:bibfield-address};
\item |year|:表示该汇编的出版年,其格式参见\ref{subsec:bibfield-year}。
\end{itemize}

\begin{note}
|author|字段和|editor|字段应该至少有一个存在,但也可同时存在。若|author|字段存在,
{\BibTeX}将使用|author|字段的值作为该汇编的主要责任者;否则,若|editor|字段存在,
{\BibTeX}将使用|editor|字段的值作为该汇编的主要责任者;若|author|字段和|editor|字段
都不存在,{\BibTeX}将使用``Anon''或``佚名''作为该汇编的主要责任者;若|author|字段
和|editor|字段都存在,{\BibTeX}将使用|author|字段的值作为该汇编的主要责任者,而忽略
|editor|字段。
\end{note}

\subsubsection{可选字段}

\begin{itemize}
\item |translator|:表示该汇编的翻译者,其格式参见\ref{subsec:bibfield-translator};
\item |series|:表示该汇编所属的丛书的标题,其格式参见\ref{subsec:bibfield-series};
\item |edition|:表示该汇编的版本,其格式参见\ref{subsec:bibfield-edition};
\item |volume|:表示该汇编所属丛书或多卷书的卷号,其格式参见\ref{subsec:bibfield-volume};
\item |pages|:表示引文在该汇编中所处的页码或页码范围,其格式参见\ref{subsec:bibfield-pages};
\item |citedate|:表示该汇编的在线版本的引用日期,其格式参见\ref{subsec:bibfield-citedate};
\item |url|:表示该汇编的在线版本的引用URL,其格式参见\ref{subsec:bibfield-url};
\item |doi|:表示该汇编的电子版的DOI编号,其格式参见\ref{subsec:bibfield-doi};
\item |language|:表示该汇编的语言,其格式参见\ref{subsec:bibfield-language};
\item 其他字段将不起作用。
\end{itemize}

\begin{note}
汇编所属丛书的标题不应出现在|title|字段中,而应填写在|series|字段中。
汇编所属多卷书的卷号也不应出现在|title|字段中,而应填写在|volume|字段中。
若该汇编的语言为中文,则必须将|language|字段设置为|zh|;否则可忽略|language|字段。
\end{note}

\subsubsection{例子}

\begin{verbatim}
@collection{engesser2009,
  title={Quantum Logic},
  series={Handbook of Quantum Logic and Quantum Structures},
  editor={Kurt Engesser and Dov M Gabbay and Daniel Lehmann},
  publisher={Elsevier},  
  year={2009},  
}

@collection{maxis1982,
  title={马克思恩格斯全集},
  author={马克思 and 恩格斯},  
  volume={44},
  address={北京},
  publisher={人民出版社},
  year={1982},
  language={zh},
}
\end{verbatim}

