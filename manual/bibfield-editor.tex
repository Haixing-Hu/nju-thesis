%%%%%%%%%%%%%%%%%%%%%%%%%%%%%%%%%%%%%%%%%%%%%%%%%%%%%%%%%%%%%%%%%%%%%%%%%%%%%%%
%%  
%% 文档类 NJU-Thesis 用户手册
%%
%% 作者:胡海星,starfish (at) gmail (dot) com
%% 项目主页: https://github.com/Haixing-Hu/nju-thesis
%%
%% This file may be distributed and/or modified under the conditions of the
%% LaTeX Project Public License, either version 1.2 of this license or (at your
%% option) any later version. The latest version of this license is in:
%%
%% http://www.latex-project.org/lppl.txt
%%
%% and version 1.2 or later is part of all distributions of LaTeX version
%% 1999/12/01 or later.
%%
%%%%%%%%%%%%%%%%%%%%%%%%%%%%%%%%%%%%%%%%%%%%%%%%%%%%%%%%%%%%%%%%%%%%%%%%%%%%%%%

\subsection{editor}\label{subsec:bibfield-editor}

|editor|字段表示文献的编辑者。其格式的具体要求与|author|字段类似。

|editor|字段和|author|字段的区别在于:|author|是所引用的文献的主要负责人,
|editor|则是所引用文献本身,或其所在的合集的编辑。例如:
\begin{itemize}
\item 如果所文献是一本书,但每个章节由不同的作者所写,而整本书由某个编辑者编辑。此时,
  \begin{itemize}
  \item 如果论文中引用的是全书,则应该将该文献作为|book|类型,其|editor|字段为编辑
    者,|author|字段不填。
  \item 如果论文中引用的是全书中由某个作者写的某个章节,则应该将文献作为
    |inbook|或|incollection|类型(取决于该章节是否有自己的标题,详见
    \ref{sec:bib-type}的解释),其|author|字段为该章节作者,|editor|字段为
    专著的编辑者。
  \end{itemize}
\item 如果文献是会议论文集,此时:
  \begin{itemize}
  \item 如果论文引用的是论文集全集,则应该将文献作为|proceedings|类型,其|editor|字段
    为该论文集的编辑,|author|字段不填;
  \item 如果论文引用的是论文集中的某篇文章,则应该将文献作为|inproceedings|类型,
    其|author|字段为该文章作者,|editor|字段为该论文集的编辑者。
  \end{itemize}
\end{itemize}

关于|editor|和|author|字段的详细解释,请参见\ref{sec:bib-type}。

