%%%%%%%%%%%%%%%%%%%%%%%%%%%%%%%%%%%%%%%%%%%%%%%%%%%%%%%%%%%%%%%%%%%%%%%%%%%%%%%
%%  
%% 文档类 NJU-Thesis 用户手册
%%
%% 作者:胡海星,starfish (at) gmail (dot) com
%% 项目主页: https://github.com/Haixing-Hu/nju-thesis
%%
%% This file may be distributed and/or modified under the conditions of the
%% LaTeX Project Public License, either version 1.2 of this license or (at your
%% option) any later version. The latest version of this license is in:
%%
%% http://www.latex-project.org/lppl.txt
%%
%% and version 1.2 or later is part of all distributions of LaTeX version
%% 1999/12/01 or later.
%%
%%%%%%%%%%%%%%%%%%%%%%%%%%%%%%%%%%%%%%%%%%%%%%%%%%%%%%%%%%%%%%%%%%%%%%%%%%%%%%%

\chapter{学位论文的排版}

%%%%%%%%%%%%%%%%%%%%%%%%%%%%%%%%%%%%%%%%%%%%%%%%%%%%%%%%%%%%%%%%%%%%%%%%%%%%%%%
\section{\LaTeX 排版}


%%%%%%%%%%%%%%%%%%%%%%%%%%%%%%%%%%%%%%%%%%%%%%%%%%%%%%%%%%%%%%%%%%%%%%%%%%%%%%%
\section{学位论文的结构}


%%%%%%%%%%%%%%%%%%%%%%%%%%%%%%%%%%%%%%%%%%%%%%%%%%%%%%%%%%%%%%%%%%%%%%%%%%%%%%%
\section{使用文档类}

\subsection{文档类参数}

{\njuthesis}文档类提供了一些选项以方便使用:

\begin{description}
\item[winfonts, linuxfonts, macfonts, adobefonts] |winfonts|选项使得文档使
  用Windows系统提供的字体;|linuxfonts|选项使得文档使用Linux系统提供的字
  体;|macfonts|选项使得文档使用Mac系统提供的字体;|adobefonts|选项使得文档使
  用Adobe提供的OTF中文字体,一般来说OTF字体的显示效果要优于ttf字体。
  默认选项是|adobefonts|。
\begin{tex}
\documentclass[winfonts]{njuthesis}
\end{tex}
  \begin{table}
    \centering\noindent\small
    \begin{tabular}[t]{ccccc}
    \toprule
          & \textbf{adobefonts} &  \textbf{winfonts} & \textbf{linuxfonts} & \textbf{macfonts} \\
    \midrule
     \textbf{宋体} & {Adobe Song Std}  & {SimSun} & {AR PL SungtiL GB} &  {STSong} \\
     \textbf{黑体} & {Adobe Heiti Std} & {SimHei} & {WenQuanYi Zen Hei Mono} &  {STHeiti} \\
     \textbf{楷书} & {Adobe Kaiti Std} & {KaiTi}  & {AR PL KaitiM GB} & {STKaiti} \\
     \textbf{仿宋体} & {Adobe Fangsong Std} & {FangSong} & {STFangSong} & {STFangSong} \\
    \bottomrule
    \end{tabular}
    \caption{默认配置下不同字体选项所使用的实际字体名称}
    \label{table:fontnames}
  \end{table}
  表\ref{table:fontnames}中列出了默认配置下使用不同字体选项时所采用的实际字体
  名称。系统中必须正确安装了相应的字体才能正常编译文档。\\
  Adobe的宋体和黑体可以在其公司网站免费下载:
  \begin{center}
  \url{http://www.adobe.com/support/downloads/detail.jsp?ftpID=4421}
  \end{center}
  楷体无免费下载,但在网上可以找到。下面的网址提供了一个打包下载的地址:
  \begin{center}
  \url{http://tinker-bot.googlecode.com/files/cfonts.tar.gz}
  \end{center}

\item[nobackinfo] 该选项用于控制是否在封面背面打印导师签名信息。如果设置了此选
  项,则不在封面背面打印导师签名信息。此选项默认不被设置,一般情况下也无需设置
  该选项。
\begin{tex}
\documentclass[winfonts,nobackinfo]{njuthesis}
\end{tex}

\item[phd, master, bachelor] 用于设置申请的学位级别。当选择|phd|时,生成南京大学博
士学位论文,包含国家图书馆格式的封面,但不包括书脊,书脊需要单独制作;选择|master|时,
生成南京大学硕士学位论文;选择|bachelor|时,生成南京大学学士学位论文。
\begin{tex}
\documentclass[winfonts,phd]{njuthesis}
\end{tex}
\begin{note}
这三个选项必须设置一个且只能设置一个。
\end{note}

\item[showcomments] 如果设置了此选项,则文档中的注释会被显示出来,否则所有注释
都会被隐藏。此选项默认不被设置。
\begin{tex}
\documentclass[winfonts,phd,showcomments]{njuthesis}
\end{tex}

\end{description}

{\njuthesis}文档类不提供对字号、字体和单双面打印的选择选项。因为国内各大学的学位
论文基本上都要求使用小四号宋体,双面打印。

\subsection{元信息}

在输入正文之前,需要在{\LaTeX}文件的导言区设置论文的元信息。{\njuthesis}文档类将根据
用户所设置的元信息自动生成诸如论文中英文封面、中英文摘要和论文出版授权书等页面。

\subsubsection{国家图书馆封面元信息设置}

本节描述论文的国家图书馆封面的元信息设置。只有博士学位论文才需要提供国家图书馆
封面。若申请的学位为硕士或学士,则可完全忽略本节所描述的命令。

\paragraph{分类号}

命令\cs{classification}用于设置论文按照《中国图书资料分类法》的分类编号。此属性
必须被设置。具体的分类号需咨询学校图书馆的老师。
\begin{tex}
  \classification{O175.2}
\end{tex}

\paragraph{密级}

命令\cs{securitylevel}设置论文的密级。论文的密级必须按照\std{GB/T 7156-2003}标准
\cite{gbt7156-2003}进行填写。

根据\std{GB/T 7156-2003}标准,文献保密等级分为$5$级,即“公开级”、“限制级”、“秘密
级”、“机密级”、“绝密级”\cite{gbt7156-2003}。本文档类中预定义了该标准中文献保密等
级的五个等级的代码常量:
\begin{description}
\item[\cs{openlevel}] 公开级:论文可在国内外发行和交换。
\item[\cs{controllevel}] 限制级:论文内容不涉及国家秘密,但在一定时间内限制
  其交流和使用范围。
\item[\cs{confidentiallevel}] 秘密级:论文内容涉及一般国家秘密。
\item[\cs{clasifiedlevel}] 机密级:论文内容涉及重要的国家秘密 。
\item[\cs{mostconfidentiallevel}] 绝密级:论文内容涉及最重要的国家秘密。
\end{description}

如果未设置\cs{securitylevel},其默认值将被设置为\cs{openlevel},即“公开级”。

\begin{tex}
\securitylevel{\controllevel}
\end{tex}

\paragraph{UDC编号}

命令\cs{udc}用于设置论文按照《国际十进分类法UDC》的分类编号。此属性可选,默认值为空白。
\begin{tex}
\udc{004.72}
\end{tex}

国际十进分类法(Universal Decimal Classification,简称UDC),又称为通用十进制分
类法,是世界上规模最大、用户最多、影响最广泛的一部文献资料分类法。自1899--1905年
比利时学者奥特勒和拉封丹共同主编、出版UDC法文第一版以来,现已有20多种语言的各种
详略版本。近百年来,UDC已被世界上几十个国家的10多万个图书馆和情报机构采用。UDC目
前已成为名符其实的国际通用文献分类法。

论文的具体UDC编号需咨询学校图书馆的老师,或在下面网址查询:
\begin{center}
\url{http://www.udcc.org/udcsummary/php/index.php?lang=chi}
\end{center}

\paragraph{论文标题及副标题}

命令\cs{nlctitlea}、\cs{nlctitleb}和\cs{nlctitlec}分别用于设置国家图书馆封面的
论文标题及副标题的第一行、第二行和第三行。其中,\cs{nlctitlea}为可选,默认值为
用户通过\cs{title}命令设置的中文标题;\cs{nlctitleb}和\cs{nlctitlec}亦为可选,
其默认值为空白。这三个命令是为了让用户在论文标题较长时手动进行分割换行。
\begin{tex}
\nlctitlea{基于小世界理论的}
\nlctitleb{数据中心网络模型研究}
\end{tex}

\begin{note}
\cs{nlctitlea}、\cs{nlctitleb}和\cs{nlctitlec}命令的参数中都不能再出现换行。
\end{note}

\paragraph{导师信息}

命令\cs{supervisorinfo}用于设置论文作者的导师的单位名称及联系地址。此属性必须被设置。
\begin{tex}
\supervisorinfo{南京大学计算机科学与技术系,南京市汉口路22号,210093}
\end{tex}

\paragraph{答辩委员会主席}

命令\cs{chairman}用于设置论文答辩委员会主席的姓名和职称。此属性必须被设置。
\begin{tex}
\chairman{王重阳\hspace{1em}教授}
\end{tex}

\paragraph{评阅人}

命令\cs{reviewera}、\cs{reviewerb}、\cs{reviewerc}、\cs{reviewerd}分别用于设置
论文的第一、第二、第三和第四评阅人的姓名和职称。这四个命令为可选,默认值为空白。
\begin{tex}
\reviewera{张三丰~~教授}
\reviewerb{张无忌~~副教授}
\reviewerc{黄裳~~教授}
\reviewerd{郭靖~~研究员}
\end{tex}

\subsubsection{中文封面元信息设置}

\paragraph{论文标题}

命令\cs{title}用于设置论文的中文标题。此属性必须被设置。
\begin{tex}
\title{基于小世界理论的数据中心网络模型}
\end{tex}
\begin{note}
\cs{title}的参数中不可换行,也不能使用\cs{thanks}脚注。
\end{note}

\paragraph{作者姓名}

命令\cs{author}用于设置论文作者的姓名。此属性必须被设置。
\begin{tex}
\author{张三}
\end{tex}
\begin{note}
\cs{author}的参数中不可换行,也不能使用\cs{thanks}脚注。
\end{note}

\paragraph{作者电话}

命令\cs{telphone}用于设置论文作者的电话号码。此属性必须被设置。
\begin{tex}
\telphone{13671413272}
\end{tex}

\paragraph{作者邮件}

命令\cs{email}用于设置论文作者的电子邮件地址。此属性必须被设置。
\begin{tex}
\email{san.zhang@gmail.com}
\end{tex}

\paragraph{作者学号}

命令\cs{studentnum}用于设置论文作者的学号。此属性必须被设置。
\begin{tex}
\studentnum{MG1033012}
\end{tex}

\paragraph{入学年份}

命令\cs{grade}用于设置论文作者的入学年份(即年级),用一个阿拉伯数字表示。此属性
必须被设置。
\begin{tex}
\grade{2012}
\end{tex}

\paragraph{导师姓名职称}

命令\cs{supervisor}用于设置论文作者的导师的姓名和职称。此属性必须被设置。
\begin{tex}
\supervisorname{李四~~教授}
\end{tex}

\paragraph{导师电话}

命令\cs{supervisortelphone}用于设置论文作者的导师的姓名和职称。此属性必须被设置。
\begin{tex}
\supervisortelphone{13671607471}
\end{tex}

\paragraph{学科专业}

命令\cs{major}用于设置论文作者的学科与专业方向。此属性必须被设置。
\begin{tex}
\major{计算机软件与理论}
\end{tex}
\begin{note}
\cs{major}的参数中不可换行。
\end{note}

\paragraph{研究方向}

命令\cs{researchfield}用于设置论文作者的研究方向。此属性必须被设置。
\begin{tex}
\major{计算机网络与信息安全}
\end{tex}
\begin{note}
\cs{researchfield}的参数中不可换行。
\end{note}

\paragraph{院系名称}

命令\cs{department}用于设置论文作者所在院系的中文名称。此属性必须被设置。
\begin{tex}
\department{计算机科学与技术系}
\end{tex}
\begin{note}
\cs{department}的参数中不可换行。
\end{note}

\paragraph{学校名称}

命令\cs{institute}用于设置论文作者所在学校或机构的名称,该学校或机构也是所申请学
位的颁发机构。此命令为可选,默认值为``南京大学''。
\begin{tex}
\institute{南京大学}
\end{tex}
\begin{note}
\cs{institute}的参数中不可换行。
\end{note}

\paragraph{提交日期}

命令\cs{submitdate}用于设置论文的提交日期,需设置年、月、日。此属性必须被设置。
\begin{tex}
\submitdate{2013年6月10日}
\end{tex}

\paragraph{答辩日期}

命令\cs{defenddate}用于设置论文的答辩日期,需设置年、月、日。此属性必须被设置。
\begin{tex}
\defenddate{2013年6月27日}
\end{tex}

\paragraph{定稿日期}

命令\cs{date}用于设置论文的定稿日期,该日期将出现在中文封面下方以及书脊下方。需设
置年、月、日。此属性可选,默认值为最后一次编译时的日期,精确到日。
\begin{tex}
\date{2013年5月27日}
\end{tex}

\subsubsection{英文封面元信息设置}

\paragraph{论文标题}

命令\cs{englishtitle}用于设置论文的英文标题。此属性必须被设置。
\begin{tex}
\englishtitle{Network Models of Data Centers based on%
 the Small World Theory}
\end{tex}
\begin{note}
\cs{englishtitle}的参数中不可换行,也不能使用\cs{thanks}脚注。
\end{note}

\paragraph{作者姓名}

命令\cs{englishauthor}用于设置论文的作者姓名的汉语拼音,此属性必须被设置。作者姓
名的汉语拼音必须遵循\std{GB/T 16159-2012}标准。
\begin{tex}
\englishauthor{Wei Xiaobao}
\end{tex}
\begin{note}
\cs{englishauthor}的参数中不可换行,也不能使用\cs{thanks}脚注。
\end{note}

\paragraph{导师姓名职称}

命令\cs{englishsupervisor}用于设置论文作者的导师姓名的汉语拼音和导师职称的英文翻译。
此属性必须被设置。导师姓名的汉语拼音必须遵循\std{GB/T 16159-2012}标准。
\begin{tex}
\englishsupervisor{Professor CHEN Jin-Nan}
\end{tex}

\paragraph{作者专业}

命令\cs{englishmajor}用于设置论文作者的学科与专业方向的英文名。此属性必须被设置。
\begin{tex}
\englishmajor{Compuer Software and Theory}
\end{tex}
\begin{note}
\cs{englishmajor}的参数中不可换行。
\end{note}

\paragraph{院系名称}

命令\cs{englishdepartment}用于设置论文作者所在院系的英文名称。此属性必须被设置。
\begin{tex}
\englishdepartment{Department of Computer Science and Technology}
\end{tex}
\begin{note}
\cs{englishdepartment}的参数中不可换行。
\end{note}

\paragraph{学校名称}

命令\cs{englishinstitute}用于设置论文作者所在学校或机构的英文名称,此学校或机构
也是所申请学位的颁发机构。此属性可选,默认值为``Nanjing University''。
\begin{tex}
\englishinstitute{Nanjing University}
\end{tex}
\begin{note}
\cs{englishinstitute}的参数中不可换行。
\end{note}

\paragraph{完成日期}

命令\cs{englishdate}用于设置论文完成日期的英文形式,它将出现在英文封面下方。需
设置年、月、日。日期格式使用美国的日期格式,即``Month day, year'',其中
``Month''为月份的英文名全称,首字母大写;``day''为该月中日期的阿拉伯数字表示;
``year''为年份的四位阿拉伯数字表示。此属性可选,默认值为最后一次编译时的日期。
\begin{tex}
\englishdate{May 1, 2013}
\end{tex}

\subsubsection{中文摘要页元信息设置}

\paragraph{标题及副标题}

命令\cs{abstracttitlea}和\cs{abstracttitleb}分别用于设置中文摘要页面的论文标题
及副标题的第一行和第二行。\cs{abstracttitlea}命令为可选,其默认值为使用\cs{title}
命令所设置的论文标题;\cs{abstracttitleb}命令为可选,其默认值为空白。这两个命令
是为了让用户在论文标题较长时手动进行分割换行。
\begin{tex}
\abstracttitlea{基于小世界理论的}
\abstracttitleb{数据中心网络模型研究}
\end{tex}
\begin{note}
\cs{abstracttitlea}和\cs{abstracttitleb}命令的参数中都不能出现换行。
\end{note}

\subsubsection{英文摘要页元信息设置}

\paragraph{标题及副标题}

命令\cs{abstracttitlea}和\cs{abstracttitleb}分别用于设置英文摘要页面的论文标题
及副标题的第一行和第二行。\cs{englishabstracttitlea}命令为可选,其默认值为使用
\cs{englishtitle}命令所设置的论文英文标题;\cs{englishabstracttitleb}命令为可
选,其默认值为空白。这两个命令是为了让用户在论文标题较长时手动进行分割换行。
\begin{tex}
\englishabstracttitlea{A Network Model of Data Centers}
\englishabstracttitleb{Based on the Small World Theory}
\end{tex}
\begin{note}
\cs{englishabstracttitlea}和\cs{englishabstracttitleb}命令的参数中都不能换行。
\end{note}

%%%%%%%%%%%%%%%%%%%%%%%%%%%%%%%%%%%%%%%%%%%%%%%%%%%%%%%%%%%%%%%%%%%%%%%%%%%%%%%
\subsection{封面}

\paragraph{生成国家图书馆封面}

命令\cs{makenlctitle}用于生成论文的国家图书馆封面。此命令必须被用在{\TeX}文档
的\cs{begin{document}}命令之后和\cs{frontmatter}命令之前。目前只有博士学位论文要
求制作国家图书馆封面,硕士学位论文和学士学位论文不需要。
\begin{tex}
\makenlctitle
\end{tex}

\paragraph{生成中文封面}

命令\cs{maketitle}用于生成论文的中文封面。此命令必须被用在{\TeX}文档的
\cs{begin{document}}命令之后和\cs{frontmatter}命令之前。
\begin{tex}
\maketitle
\end{tex}

\paragraph{生成英文封面}

命令\cs{makeenglishtitle}用于生成论文的英文封面。此命令必须被用在{\TeX}文档的
\cs{begin{document}}命令之后和\cs{frontmatter}命令之前。
\begin{tex}
\makeenglishtitle
\end{tex}

%%%%%%%%%%%%%%%%%%%%%%%%%%%%%%%%%%%%%%%%%%%%%%%%%%%%%%%%%%%%%%%%%%%%%%%%%%%%%%%
\subsection{摘要}

\paragraph{中文摘要}

\env{abstract}为中文摘要环境。此环境必须被用在{\TeX}文档的\cs{frontmatter}命令之后和
\cs{mainmatter}命令之前。
\begin{tex}
\begin{abstract}
本文基于小世界理论,研究了数据中心的网络模型。………………
\end{abstract}
\end{tex}

\paragraph{中文关键词}

命令\cs{keywords}用于设置中文关键词。此命令必须被用在\env{abstract}环境中。关键词
之间用中文全角分号隔开。
\begin{tex}
\begin{abstract}
本文基于小世界理论,研究了数据中心的网络模型。………………
\keywords{数据中心;网络模型;小世界理论}
\end{abstract}
\end{tex}

\paragraph{英文摘要}

\env{englishabstract}为英文摘要环境。此环境必须被用在{\TeX}文档的
\env{abstract}环境之后和\cs{mainmatter}命令之前。
\begin{tex}
\begin{englishabstract}
In this paper, we studied the network model of data centers,
based on the theory of small worlds. ....
\end{englishabstract}
\end{tex}

\paragraph{英文关键词}

命令\cs{englishkeywords}用于设置英文关键词。此命令必须被用在\env{englishabstract}
环境中。关键词之间用英文半角逗号隔开。
\begin{tex}
\begin{englishabstract}
In this paper, we studied the network model of data centers,
based on the theory of small worlds. ....
\englishkeywords{Data Center, Network Model, Small World}
\end{englishabstract}
\end{tex}

%%%%%%%%%%%%%%%%%%%%%%%%%%%%%%%%%%%%%%%%%%%%%%%%%%%%%%%%%%%%%%%%%%%%%%%%%%%%%%%

\subsection{前言}

\env{preface}为论文前言环境。此环境必须被用在{\TeX}文档的\env{englishabstract}环
境之后和\cs{tableofcontents}命令之前。
\begin{tex}
\begin{preface}
 复杂网络的研究可上溯到20世纪60年代对ER网络的研究。90年后代随着
 Internet的发展,以及对人类社会、通信网络、生物网络、社交网络等
 各领域研究的深入,发现了小世界网络和无尺度现象等普适现象与方法。
 对复杂网络的定性定量的科学理解和分析,已成为如今网络时代科学研究
 的一个重点课题。

 在此背景下,由于云计算时代的到来,本文针对面向云计算的数据中心网
 络基础设施设计中的若干问题,进行了几方面的研究。本文的创造性研究
 成果主要如下几方面:

 ………

 \vspace{1cm}
 \begin{flushright}
  韦小宝\\
  2013年夏于南京大学南苑
 \end{flushright}
\end{preface}
\end{tex}

%%%%%%%%%%%%%%%%%%%%%%%%%%%%%%%%%%%%%%%%%%%%%%%%%%%%%%%%%%%%%%%%%%%%%%%%%%%%%%%

\subsection{目录}

\paragraph{章节目录}

命令\cs{tableofcontents}用于生成论文章节目录。此命令必须被用在{\TeX}文档的
\env{preface}环境之后和\cs{mainmatter}命令之前。
\begin{tex}
\tableofcontents
\end{tex}

\paragraph{表格目录}

命令\cs{listoftables}用于生成论文表格目录。此命令为可选命令。此命令必须被用在
{\TeX}文档的\cs{tableofcontents}命令之后和\cs{mainmatter}命令之前。
\begin{tex}
\listoftables
\end{tex}

\paragraph{插图目录}

命令\cs{listoffigures}用于生成论文插图目录。此命令为可选命令。此命令必须被用在
{\TeX}文档的\cs{tableofcontents}命令之后和\cs{mainmatter}命令之前。
\begin{tex}
\listoffigures
\end{tex}

%%%%%%%%%%%%%%%%%%%%%%%%%%%%%%%%%%%%%%%%%%%%%%%%%%%%%%%%%%%%%%%%%%%%%%%%%%%%%%%

%% \subsection{缩写和符号列表}

%%%%%%%%%%%%%%%%%%%%%%%%%%%%%%%%%%%%%%%%%%%%%%%%%%%%%%%%%%%%%%%%%%%%%%%%%%%%%%%

%% \subsection{术语表}

%%%%%%%%%%%%%%%%%%%%%%%%%%%%%%%%%%%%%%%%%%%%%%%%%%%%%%%%%%%%%%%%%%%%%%%%%%%%%%%

\subsection{正文}

%%%%%%%%%%%%%%%%%%%%%%%%%%%%%%%%%%%%%%%%%%%%%%%%%%%%%%%%%%%%%%%%%%%%%%%%%%%%%%%
\subsubsection{章、节}

%%%%%%%%%%%%%%%%%%%%%%%%%%%%%%%%%%%%%%%%%%%%%%%%%%%%%%%%%%%%%%%%%%%%%%%%%%%%%%%
\subsubsection{插图}

%%%%%%%%%%%%%%%%%%%%%%%%%%%%%%%%%%%%%%%%%%%%%%%%%%%%%%%%%%%%%%%%%%%%%%%%%%%%%%%
\subsubsection{表格}

%%%%%%%%%%%%%%%%%%%%%%%%%%%%%%%%%%%%%%%%%%%%%%%%%%%%%%%%%%%%%%%%%%%%%%%%%%%%%%%
\subsubsection{公式}

%%%%%%%%%%%%%%%%%%%%%%%%%%%%%%%%%%%%%%%%%%%%%%%%%%%%%%%%%%%%%%%%%%%%%%%%%%%%%%%
\subsubsection{定理}

%%%%%%%%%%%%%%%%%%%%%%%%%%%%%%%%%%%%%%%%%%%%%%%%%%%%%%%%%%%%%%%%%%%%%%%%%%%%%%%
\subsubsection{证明}

%%%%%%%%%%%%%%%%%%%%%%%%%%%%%%%%%%%%%%%%%%%%%%%%%%%%%%%%%%%%%%%%%%%%%%%%%%%%%%%
\subsubsection{算法}

%%%%%%%%%%%%%%%%%%%%%%%%%%%%%%%%%%%%%%%%%%%%%%%%%%%%%%%%%%%%%%%%%%%%%%%%%%%%%%%
\subsubsection{交叉引用}

%%%%%%%%%%%%%%%%%%%%%%%%%%%%%%%%%%%%%%%%%%%%%%%%%%%%%%%%%%%%%%%%%%%%%%%%%%%%%%%
\subsubsection{脚注}

%%%%%%%%%%%%%%%%%%%%%%%%%%%%%%%%%%%%%%%%%%%%%%%%%%%%%%%%%%%%%%%%%%%%%%%%%%%%%%%
\subsubsection{引文}

%%%%%%%%%%%%%%%%%%%%%%%%%%%%%%%%%%%%%%%%%%%%%%%%%%%%%%%%%%%%%%%%%%%%%%%%%%%%%%%

\subsection{致谢}

\env{acknowledgement}环境用于生成致谢章节。此环境必须被用在论文的最后一章(通
常是“结论”章节)之后以及{\TeX}文档的\cs{appendix}命令和\cs{backmatter}命令之前。

\begin{tex}
\begin{acknowledgement}
首先感谢我的母亲韦春花对我的支持。其次感谢我的导师陈近南对我的
精心指导和热心帮助。接下来,感谢我的师兄茅十八和风际中,他们阅
读了我的论文草稿并提出了很有价值的修改建议。

最后,感谢我亲爱的老婆们:双儿、苏荃、阿珂、沐剑屏、曾柔、建宁
公主、方怡,感谢你们在生活上对我无微不至的关怀和照顾。我爱你们!
\end{acknowledgement}
\end{tex}

%%%%%%%%%%%%%%%%%%%%%%%%%%%%%%%%%%%%%%%%%%%%%%%%%%%%%%%%%%%%%%%%%%%%%%%%%%%%%%%
\subsection{参考文献列表}

%%%%%%%%%%%%%%%%%%%%%%%%%%%%%%%%%%%%%%%%%%%%%%%%%%%%%%%%%%%%%%%%%%%%%%%%%%%%%%%
\subsection{附录}

%%%%%%%%%%%%%%%%%%%%%%%%%%%%%%%%%%%%%%%%%%%%%%%%%%%%%%%%%%%%%%%%%%%%%%%%%%%%%%%
\subsection{索引}

%%%%%%%%%%%%%%%%%%%%%%%%%%%%%%%%%%%%%%%%%%%%%%%%%%%%%%%%%%%%%%%%%%%%%%%%%%%%%%%
\subsection{作者简历与科研成果}

\env{resume}环境用于生成致谢章节。此环境必须被放在{\TeX}文档的\cs{backmatter}
命令之后。

\env{authorinfo}环境用于生成论文作者简介。

\env{education}环境用于生成论文作者教育经历列表。

\env{publications}环境用于生成论文作者在攻读学位期间发表的论文的列表。
\env{projects}环境用于生成论文作者在攻读学位期间参与的科研课题的列表。

\begin{tex}
\begin{resume}
% 论文作者身份简介,一句话即可。
\begin{authorinfo}
\noindent 韦小宝,男,汉族,1985年11月出生,江苏省扬州人。
\end{authorinfo}
% 论文作者教育经历列表,按日期从近到远排列,不包括将要申请的学位。
\begin{education}
\item[2007.9 --- 2010.6] 南京大学计算机科学与技术系 \hfill 硕士
\item[2003.9 --- 2007.6] 南京大学计算机科学与技术系 \hfill 本科
\end{education}
% 论文作者在攻读学位期间所发表的文章的列表,按发表日期从近到远排列。
\begin{publications}
\item Xiaobao Wei, Jinnan Chen, ``Voting-on-Grid Clustering 
  for Secure Localization in Wireless Sensor Networks,'' in 
  \textsl{Proc. IEEE International Conference on Communications
   (ICC) 2010}, May. 2010.
\item Xiaobao Wei, Shiba Mao, Jinnan Chen, ``Protecting Source
  Location Privacy in Wireless Sensor Networks with Data 
  Aggregation,'' in \textsl{Proc. 6th International Conference 
  on Ubiquitous Intelligence and Computing (UIC) 2009}, Oct.
   2009.
\end{publications}
% 论文作者在攻读学位期间参与的科研课题的列表,按照日期从近到远排列。
\begin{projects}
\item 国家自然科学基金面上项目``无线传感器网络在知识获取过程中的若干安
 全问题研究''(课题年限~2010.1 --- 2012.12),负责位置相关安全问题的
 研究。
\item 江苏省知识创新工程重要方向项目下属课题``下一代移动通信安全机制研
 究''(课题年限~2010.1 --- 2010.12),负责LTE/SAE认证相关的安全问题
 研究。
\end{projects}
\end{resume}
\end{tex}

%%%%%%%%%%%%%%%%%%%%%%%%%%%%%%%%%%%%%%%%%%%%%%%%%%%%%%%%%%%%%%%%%%%%%%%%%%%%%%%

\subsection{学位论文出版授权书}

命令\cs{makelicense}用于生成《学位论文出版授权书》。该授权书中的一些字段将根据
用户所设置的文档属性自动填写,其他字段需由作者将论文打印出来后用笔手工填写。

此命令应该用于{\TeX}文档的\cs{end{document}}命令之前。

\begin{tex}
\makelicense
\end{tex}
