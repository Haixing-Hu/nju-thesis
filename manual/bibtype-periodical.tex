%%%%%%%%%%%%%%%%%%%%%%%%%%%%%%%%%%%%%%%%%%%%%%%%%%%%%%%%%%%%%%%%%%%%%%%%%%%%%%%
%%  
%% 文档类 NJU-Thesis 用户手册
%%
%% 作者:胡海星,starfish (at) gmail (dot) com
%% 项目主页: https://github.com/Haixing-Hu/nju-thesis
%%
%% This file may be distributed and/or modified under the conditions of the
%% LaTeX Project Public License, either version 1.2 of this license or (at your
%% option) any later version. The latest version of this license is in:
%%
%% http://www.latex-project.org/lppl.txt
%%
%% and version 1.2 or later is part of all distributions of LaTeX version
%% 1999/12/01 or later.
%%
%%%%%%%%%%%%%%%%%%%%%%%%%%%%%%%%%%%%%%%%%%%%%%%%%%%%%%%%%%%%%%%%%%%%%%%%%%%%%%%

\subsection{期刊}\label{subsec:bibtype-periodical}

期刊是连续出版物的一种。它是指一种载有卷期号或年月顺序号、计划无限期地连续出版发行的出版
物\cite{gbt7714-2005}。它所对应的{\BibTeX}文献项类型为|periodical|;对应的
\std{GB/T 3469-1983}文献类型单字码为|J|\cite{gbt3469-1983}。

|periodical|不是{\BibTeX}的标准类型,而是{\njuthesis}的扩展。之所以用|periodical|这
个名称而不是|journal|,是因为|journal|已经是{\BibTeX}的标准字段名,为避免重名冲突,
所以只能叫|periodical|。

\begin{note}
如果引文是期刊的某一期或某几期,则应使用|periodical|类型;如果引文是期刊中的文章,请使
用|article|类型,具体参见\ref{subsec:bibtype-article}。
杂志属于期刊的一种,但报纸和期刊有区别,因为他们对应的\std{GB/T 3469-1983}文献类型单
字码不同。报纸应采用|newspaper|文献项类型,参见\ref{subsec:bibtype-newspaper}。
\end{note}

\subsubsection{必需字段}

\begin{itemize}
\item |title|:表示该期刊的标题,其格式参见\ref{subsec:bibfield-title};
\item |publisher|:表示该期刊的出版社,其格式参见\ref{subsec:bibfield-publisher};
\item |address|:表示该期刊的出版地,其格式参见\ref{subsec:bibfield-address};
\item |year|表示所引用的期刊的出版年,或所引用的一系列连续期刊的出版年范围,其格式
  参见\ref{subsec:bibfield-year}。
\end{itemize}

\subsubsection{可选字段}

\begin{itemize}
\item |author|:表示该期刊的发行机构或编辑,其格式参见\ref{subsec:bibfield-author};
\item |volume|: 表示所引用的期刊的卷号,或所引用的一系列连续期刊的卷号范围,其格式参
  见\ref{subsec:bibfield-volume};
\item |number|: 表示所引用的期刊的期号,或所引用的一系列连续期刊的期号范围,其格式参
  见\ref{subsec:bibfield-number};
\item |citedate|:表示所引用的期刊的在线版本的引用日期,其格式参见\ref{subsec:bibfield-citedate};
\item |url|:表示该期刊的在线版本的引用URL,其格式参见\ref{subsec:bibfield-url};
\item |doi|:表示该期刊的电子版的DOI编号,其格式参见\ref{subsec:bibfield-doi};
\item |language|:表示该期刊的语言,其格式参见\ref{subsec:bibfield-language};
\item 其他字段将不起作用。
\end{itemize}


\begin{note}
若该期刊对应的文献项有|url|或|doi|字段,则其必须也有|citedate|字段。
若该期刊的语言为中文,则必须将|language|字段设置为|zh|;否则可忽略|language|字段。
\end{note}

\subsubsection{例子}

\begin{verbatim}
@periodical{science1983,
  author={{American Association for the Advancement of Science}},
  title={Science},
  year={1883},
  volume={1},
  number={1},
  address={Washington},
  publisher={American Association for the Advancement of Science},
}

@periodical{dizhipinlun1936,
  author={{中国地质学会}},
  title={地质论评},
  year={1936},
  volume={1},
  number={1},
  address={北京},
  publisher={地质出版社},
  language={zh},
}

@periodical{tushuguanxutongxun1,
  author={{中国图书馆学会}},
  title={图书馆学通讯},
  year={1957-1990},
  number={1-4},
  address={北京},
  publisher={北京图书馆},
  language={zh},
}

@periodical{tushuguanxuetongxun17,
  author={{中国图书馆学会}},
  title={图书馆学通讯},
  year={1957-1990},
  volume={17-57},
  address={北京},
  publisher={北京图书馆},
  language={zh},
}
\end{verbatim}

