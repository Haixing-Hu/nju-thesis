%%%%%%%%%%%%%%%%%%%%%%%%%%%%%%%%%%%%%%%%%%%%%%%%%%%%%%%%%%%%%%%%%%%%%%%%%%%%%%%
%%  
%% 文档类 NJU-Thesis 用户手册
%%
%% 作者:胡海星,starfish (at) gmail (dot) com
%% 项目主页: https://github.com/Haixing-Hu/nju-thesis
%%
%% This file may be distributed and/or modified under the conditions of the
%% LaTeX Project Public License, either version 1.2 of this license or (at your
%% option) any later version. The latest version of this license is in:
%%
%% http://www.latex-project.org/lppl.txt
%%
%% and version 1.2 or later is part of all distributions of LaTeX version
%% 1999/12/01 or later.
%%
%%%%%%%%%%%%%%%%%%%%%%%%%%%%%%%%%%%%%%%%%%%%%%%%%%%%%%%%%%%%%%%%%%%%%%%%%%%%%%%

\subsection{书籍中的析出文献}\label{subsec:bibtype-inbook}

书籍中的析出文献是指书籍中某一具有独立标题和作者的章节。它所对应的{\BibTeX}文献
项类型为|inbook|;所对应的\std{GB/T 3469-1983}文献类型单字码为|M|\cite{gbt3469-1983}。

\begin{note}
如果是汇编中的析出文献,请使用|incollection|类型,具体参见\ref{subsec:bibtype-incollection}。
这和标准{\BibTeX}中对|inbook|与|incollection|类型的定义有所不同。
\end{note}

\subsubsection{必需字段}

\begin{itemize}
\item |author|:表示析出文献的作者,其格式参见\ref{subsec:bibfield-author}。
\item |title|:表示析出文献的标题,其格式参见\ref{subsec:bibfield-title}。
\item |booktitle|:表示书籍的标题,其格式参见\ref{subsec:bibfield-title}。
\item |editor|:表示书籍的编辑者或作者,其格式参见\ref{subsec:bibfield-editor}。
\item |address|:表示书籍的出版地,其格式参见\ref{subsec:bibfield-address}。如果
  该字段不存在,{\BibTeX}排版时将会用``[S.l.]''或``[出版地不详]''替代。
\item |publisher|:表示书籍的出版者,其格式参见\ref{subsec:bibfield-publisher}。
  如果该字段不存在,{\BibTeX}排版时将会用``[s.n.]''或``[出版者不详]''替代。
\item |year|:表示书籍的出版年,其格式参见\ref{subsec:bibfield-year}。
\item |pages|:表示析出文献在书籍中所处的页码或页码范围,其格式参见\ref{subsec:bibfield-pages}。
\end{itemize}

\subsubsection{可选字段}

\begin{itemize}
\item |translator|:表示析出文献的翻译者,其格式参见\ref{subsec:translator}。
\item |edition|:表示专著的版本,其格式参见\ref{subsec:bibfield-edition}。
\item |citedate|:表示析出文献的在线版本的引用日期,其格式参见\ref{subsec:bibfield-citedate}。
\item |url|:表示析出文献的在线版本的引用URL,其格式参见\ref{subsec:bibfield-url}。
\item |doi|:表示析出文献的DOI编码,其格式参见\ref{subsec:bibfield-doi}
\item |language|:表示专著的语言,其格式参见\ref{subsec:bibfield-language}。若语
  言为中文,此项必须填|zh|;否则,此项可省略。
\item 其他字段将不起作用。
\end{itemize}

